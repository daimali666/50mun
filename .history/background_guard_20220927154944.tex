\documentclass{article}
\usepackage{ctex}
\usepackage{geometry,ragged2e,setspace,graphicx,verbatim}
\begin{document}
    \newgeometry{left=2.5cm,bottom=0cm}
    \begin{titlepage}
    \begin{center}
        \setstretch{2}
        \includegraphics{./50mun.png}\\
        \begin{flushleft}
            \vspace*{3em}
            \normalfont{\zihao{1}\CJKfamily{hei}\bfseries 合肥市五十中第十届模拟联合国大会}\\
            \normalfont{\bfseries \Large\textrm{The 10th Model United Nations of Hefei No. 50 Middle School}}\\
            \vspace*{6em}
            \centering\normalfont{\zihao{0}\CJKfamily{hei}\bfseries 背景文件}
            \\
            \vspace*{3em}
            \centering\normalfont{\Huge\bfseries \textrm{Background Guide}}\\
            \vspace*{5em}
            \centering\normalfont{\zihao{0}\CJKfamily{hei}\bfseries 联合国安全理事会}
            \\
            \vspace*{3em}
            \centering\normalfont{\Huge\bfseries \textrm{United Nations Security Council }}\\
            \vspace*{3em}
            \centering\normalfont{\Huge\bfseries\CJKfamily{hei} 议题:第四次中东战争的调停与善后}\\
            \vspace*{3em}
            \centering\normalfont{\Huge\bfseries\CJKfamily{hei} 作者:模拟联合国社团学术部}
        \end{flushleft}
    \end{center}
    \end{titlepage}
    \clearpage
    \tableofcontents
    \clearpage
    \section{委员会致辞}
    \clearpage
    \section{议题介绍}
        本次会议的议题是第四次中东战争的调停与善后,包括调停和善后两个方面。历史上1973年10月22日早上,联合国通过了停火决议。本次会议背景设置在停火之前,代表必须先撰写决议草案停战。决议草案的内容包括强制停战撤军、划定联合国管理区域等。该部分预计在第一议程完成。

        第二部分是善后,可以包括多方面,如难民、能源问题等。具体可以参考临时议程书。

        下面会介绍第四次中东战争的历史与现状。我们将会从第一次世界大战中东格局初步形成,讲到第四次中东战争以色列转败为胜。
        \subsection{历史与现实}
            \subsubsection{第一次世界大战:强弩之末的帝国、现代中东的形成}
                \subsubsection{战前中东}
一战前的中东版图和现在大不相同,以色列、约旦、叙利亚、沙特阿拉伯等大部分区域处在奥斯曼帝国的统治之下。虽然它们名义上属于奥斯曼帝国统治,但都被英国渗透。如波斯湾沿岸的一些阿拉伯小邦,都任由英国摆布;塞浦路斯和埃及名义上属于奥斯曼帝国,但实际上由英国实际占领和管理。1907年,英国与沙俄签订了《英俄协约》,将阿富汗划进了“势力范围”这让奥斯曼帝国的安全受到威胁,独立地位岌岌可危。

奥斯曼帝国的科技同样和现代世界格格不入。在它的首都,也就是最发达的地区伊斯坦布尔,电灯于1912年才被引进,下水道工程也才刚刚开工。

1908年,青年土耳其党革命开始,试图推翻帝国。在苏丹阿卜杜勒·哈米德二世统治期间,他废除了宪法,解散了议会,试图加强自己的独裁统治。帝国的政治活动被迫转入地下,许多秘密社团应运而生,青年土耳其党就是其中之一。它的初衷是保持帝国领土完整,摧毁独裁专制制度,恢复宪法和议会。虽然苏丹竭力派出警察部队摧毁社团,但也有一些边疆地区无法管控,比如萨洛尼卡,青年土耳其党在这里趁机发展壮大。1908年的一天\footnote{具体日期似乎不可考}青年土耳其党发动了政变,夺取了萨洛尼卡市。1909年发动资产阶级革命,推翻了阿卜杜勒·哈米德二世的独裁专制制度,建立了君主立宪制。

但青年土耳其党仍没能拯救奥斯曼帝国,1908年,名义上属于奥斯曼帝国的波黑被奥匈帝国吞并;1911年,意大利夺取了利比亚沿岸地区;1912年,在巴尔干战争中巴尔干同盟击败奥斯曼帝国,夺取了奥斯曼帝国大部分欧洲领土......青年
\end{document}