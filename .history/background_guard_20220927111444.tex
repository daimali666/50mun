\documentclass{article}
\usepackage{ctex}
\usepackage{geometry,ragged2e,setspace,graphicx,verbatim}
\begin{document}
    \newgeometry{left=2.5cm,bottom=0cm}
    \begin{titlepage}
    \begin{center}
        \setstretch{2}
        \includegraphics{./50mun.png}\\
        \begin{flushleft}
            \vspace*{3em}
            \normalfont{\zihao{1}\CJKfamily{hei}\bfseries 合肥市五十中第十届模拟联合国大会}\\
            \normalfont{\bfseries \Large\textrm{The 10th Model United Nations of Hefei No. 50 Middle School}}\\
            \vspace*{6em}
            \centering\normalfont{\zihao{0}\CJKfamily{hei}\bfseries 背景文件}
            \\
            \vspace*{3em}
            \centering\normalfont{\Huge\bfseries \textrm{Background Guide}}\\
            \vspace*{5em}
            \centering\normalfont{\zihao{0}\CJKfamily{hei}\bfseries 联合国安全理事会}
            \\
            \vspace*{3em}
            \centering\normalfont{\Huge\bfseries \textrm{United Nations Security Council }}\\
            \vspace*{3em}
            \centering\normalfont{\Huge\bfseries\CJKfamily{hei} 议题:第四次中东战争的调停与善后}\\
            \vspace*{3em}
            \centering\normalfont{\Huge\bfseries\CJKfamily{hei} 作者:模拟联合国社团学术部}
        \end{flushleft}
    \end{center}
    \end{titlepage}
    \clearpage
    \tableofcontents
    \clearpage
    \section{委员会致辞}
    \clearpage
    \section{议题介绍}
        本次会议的议题是第四次中东战争的调停与善后,包括调停和善后两个方面。历史上1973年10月22日早上,联合国通过了停火决议。本次会议背景设置在停火之前,代表必须先撰写决议草案停战。决议草案的内容包括强制停战、划定联合国g
\end{document}